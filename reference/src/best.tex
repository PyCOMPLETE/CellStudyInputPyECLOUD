
It is explained how the best estimates for the input parameters mentioned in Sec.~\ref{sec:input} are defined based on the previously mentioned papers.
Only some of them take photon induced scrubbing into account.
The notation used in the following pages, see also Eq.~(\ref{eq:not1}-\ref{eq:yy}):
\begin{center}
    \begin{tabular}{cc}
        $Y_i$ ($Y_i^*$) & Photoelectrons per impinging (absorbed) photon on first impact \\
        $Y_r$ ($Y_r^*$)& Photoelectrons per impinging (absorbed) photon after initial reflection \\
        $R_i$ & Reflection rate of photons on first impact point. \\
        $R_r$ & Reflection rate of photons that have been reflected once. \\
        $N_i$ & Photoelectrons emitted on first impact of photons. \\
        $N_r$ & Photoelectrons emitted elsewhere. \\
        $N_t$ & Total number of emitted photoelectrons.\\
        $n_\gamma$ & Number of photons emitted per proton and m in a bending magnet.
    \end{tabular}
\end{center}


\subsection{Retrieving k\_pe\_st and refl\_frac from published experiment results}

In case of uniform material properties in the beam pipe, k\_pe\_st is computed as follows:
\begin{align}
    \textbf{k\_pe\_st} = n_\gamma \frac{Y}{1-R} = N_\gamma Y^*
\end{align}
This is because all photons are eventually absorbed, either on direct impact or after an arbitrary number of reflections, and the material properties are uniform.
\begin{align}
    \textbf{refl\_frac} = \frac{N_r}{N_i+N_r} = \frac{R_iY^*_r}{(1-R_i)Y_i^* + R_iY^*_r}
\end{align}
However in the case of the sawtooth configuration in the LHC beam screen, the material properties are different at the where the synchrotron radiation impacts first.
This means that the reflectivity at this point plays a role, and the number of photons absorbed at the impact point and after initial reflection have to be weighted with the photoelectron yields $Y^*$.
\begin{align}
    \textbf{k\_pe\_st} &= N_i + N_r
    \\
    &= N_\gamma \left( (1-R_i)Y_i^* + R_iY^*_r \right)
    \\
    &= N_\gamma \left( Y_i + R_iY^*_r \right)
    \label{eq:parts}
\end{align}
One simplification is that photons are not reflected back to the sawtooth material, given its relatively small portion relative to the total circumference.
This corresponds to $R_r=0$.
Quantitatively, k\_pe\_st and refl\_frac are assessed in Sec.~\ref{sec:parameters}.

For \textbf{alimit}, the critical angle of the synchrotron radiation from Eq.~(\ref{eq:crit_angle}) could be used for $\omega = W_\text{Cu}/\hbar$: 0.36~mrad.
For \textbf{e\_pe\_sigma} and \textbf{e\_pe\_max}, describing the kinetic energy distribution of new electron macroparticles, 5 and 7~eV have been chosen in past simulations.
As stated before, the fact that photoelectrons are immediately being accelerated by the positively charged proton beam, means that their kinetic energy is of minor influence.

For the angular distribution of photoelectrons that are generated by reflected photons
\\
(\textbf{inv\_CDF\_ refl\_photoem\_file}), a uniform distribution should be used while a realistic distribution is not available.
This approach should be better than using the cosine distribution, since the probability of multiple reflections is much higher than single reflection upon impact, since the reflectivity was measured to be larger than 80\% for the non-sawtooth material.
A 2D cosine distribution should in any case be avoided, see Eq.~(\ref{eq:angles}).


\subsection{Consistency of different measurements}

The different published experimental results on photoelectron yields and reflectivities are compared in this table.
If two values are stated for a photoelectron yield, they correspond to the measurements before and after photon scrubbing.
The reflectivities colored in red only include the forward reflectivity.
The yields in blue were not published but could be retrieved with the simple relation between $R$, $Y$ and $Y^*$ from Eq.~(\ref{eq:yy}).
\begin{center}
    \begin{tabular}{c|ccc|ccc}
        \multirow{2}{*}{Source} & \multicolumn{3}{c|}{Cu co-lam.} & \multicolumn{3}{|c}{with sawtooth} \\
        & $R$ [\%] & $Y$ [e/ph]& $Y^*$ [e/ph]& $R$ [\%]& $Y$ [e/ph]& $Y^*$ [e/ph] \\\hline
        Baglin 1998 & {\color{red}80.9} & {\color{blue}0.022} & 0.114 & {\color{red}1.8} & {\color{blue} 0.052} & 0.053 \\
        Cimino 1999 & - & 0.103/0.063 & - & - & - & - \\
        Baglin 2001 & - & - & - & {\color{red}8} & {\color{blue}0.021/0.011} & 0.029/0.015 \\
        Mahne 2004 & 82 & - & - & 10 & - & - \\
    \end{tabular}
\end{center}

The differences between the Cimino and Baglin results can be explained with different "as-received" samples of Cu.
Since only the two Baglin papers include results for sawtooth materials, these should be used for the simulation.
The reflectivities from the Mahne paper should be used because of the superior measurement apparatus for reflected photons.

\subsection{Parameters to use for the simulations}
\label{sec:parameters}
For a \textbf{conservative estimate}, the following table uses the high reflectivities from the Mahne paper and the high photoelectron yields $Y$ from the first Baglin paper, retrieved from the values for $Y^*$ as published in \cite{baglin}.
%The notation is the one used in Eq.~(\ref{eq:k}), with the subscripts $i$ and $r$ meaning "on impact" and "after reflection".

\begin{tabular}{c|cccc|cc}
Chamber type & $R_i$ & $R_r$ & $Y_i$ & $Y_r$ & $Y_i^*$ & $Y_r^*$  \\ \hline 
Cu co-lam. with sawtooth &10.0 &82.0 &5.2e-02 &2.2e-02& 5.8e-02 &1.2e-01 \\
Cu co-lam. &82.0 &82.0 &2.3e-02 &2.3e-02& 1.3e-01 &1.3e-01 \\
%	chamber & $R_i$ & $R_r$ & $Y_i$ & $Y_r$ & $Y_i^*$ & $Y_r^*$ & $N_i$ & $N_r$ & $N_t$ & $N_\gamma$ & relf\_frac & k\_pe\_st \\ \hline 
%	uniform &82.0 &82.0 &2.3e-02 &2.3e-02 &1.3e-01 &1.3e-01 &2.3e-02 &1.0e-01 &1.3e-01 &1.1e-02 &8.20e-01 &1.38e-03\\
%	with sawtooth &10.0 &82.0 &5.2e-02 &2.2e-02 &5.8e-02 &1.2e-01 &5.2e-02 &1.2e-02 &6.4e-02 &1.1e-02 &1.89e-01 &7.00e-04\\
\end{tabular}

These yields and reflectivities, together with the number of photons from Eq.~(\ref{eq:ngamma}), finally lead to the needed parameters refl\_frac and k\_pe\_st.

\begin{tabular}{c|ccc|c|cc}
Chamber type & $N_i$ & $N_r$ & $N_t$ & $n_\gamma$ & refl\_frac & k\_pe\_st \\ \hline 
Cu co-lam. with sawtooth& 5.2e-02 &1.2e-02 &6.4e-02 &1.1e-02 &1.89e-01 &7.00e-04\\
Cu co-lam.& 2.3e-02 &1.0e-01 &1.3e-01 &1.1e-02 &8.20e-01 &1.38e-03\\
\end{tabular}


A \textbf{realistic estimate} would include scrubbing effects and a much lower yield, as measured in the second Baglin paper.
With respect to the first, the yield of the sawtooth material is by a factor of $0.052/0.011 \approx 4.7$ lower.
If this factor is also applied to the yield of the other material, the following input parameters should be used.
In practise, mostly the value for $N_r$ is relevant as it denotes the amount of photoelectrons that contributes to the stripes in e-cloud simulations for dipoles and quadrupoles.

\begin{tabular}{c|cccc|cc}
Chamber type & $R_i$ & $R_r$ & $Y_i$ & $Y_r$ & $Y_i^*$ & $Y_r^*$  \\ \hline 
Cu co-lam. with sawtooth& 10.0 &82.0 &1.0e-02 &4.6e-03& 1.1e-02 &2.6e-02 \\
Cu co-lam.& 82.0 &82.0 &4.6e-03 &4.6e-03& 2.6e-02 &2.6e-02 \\
%	chamber & $R_i$ & $R_r$ & $Y_i$ & $Y_r$ & $Y_i^*$ & $Y_r^*$ & $N_i$ & $N_r$ & $N_t$ & $N_\gamma$ & relf\_frac & k\_pe\_st \\ \hline 
%	with sawtooth &10.0 &82.0 &1.0e-02 &6.2e-03 &1.1e-02 &3.5e-02 &1.0e-02 &3.5e-03 &1.4e-02 &1.1e-02 &2.55e-01 &1.48e-04\\
\end{tabular}

\begin{tabular}{c|ccc|c|cc}
Chamber type & $N_i$ & $N_r$ & $N_t$ & $n_\gamma$ & refl\_frac & k\_pe\_st \\ \hline 
Cu co-lam. with sawtooth& 1.0e-02 &2.6e-03 &1.3e-02 &1.1e-02 &2.03e-01 &1.39e-04\\
Cu co-lam.& 4.6e-03 &2.1e-02 &2.6e-02 &1.1e-02 &8.20e-01 &2.81e-04\\
\end{tabular}




%\subsection{Excluding scrubbing effects}
%
%The results from Sec.~\ref{sec:Baglin}, as shown in Sec.~\ref{sec:Mahne}, underestimate the reflectivities.
%According to Eq.~(\ref{eq:yy}), this also results in too low values for $Y^*$.
%Therefore, the yields have been corrected:
%\begin{align}
%	Y^*_\text{new} = Y^*_\text{old} \frac{1-R_\text{old}}{1-R_\text{new}}
%    \label{eq:}
%\end{align}
%
%The resulting vaues for refl\_frac and k\_pe\_st are stated in Tab.~\ref{tab:input_table}, together with intermediate results.
%"Cu co-lam. with sawtooth" corresponds to an LHC chamber with sawtooth structure only at the impact point of the photoelectrons, as it is the case for the majority of the installed material.
%
%\begin{table}[tbh]
%    \centering
%        \tiny
%    \begin{tabular}{ccccccccccccc}
%	%Type & $R_i$ & $R_r$ & $Y_i$ & $Y_r$ & $Y_i^*$ & $Y_r^*$ & $N_i$ & $N_r$ & $N_t$ & $N_\gamma$ & $R$ & k\_pe\_st \\ \hline 
%    Type & {\color{blue}$R_i$} & {\color{blue}$R_r$} & {\color{red}$Y_i$} & {\color{red}$Y_r$} & $Y_i^*$ & $Y_r^*$ & $N_i$ & $N_r$ & $N_t$ & $N_\gamma$ & refl\_frac & k\_pe\_st \\ \hline 
%	Cu co-lam. with sawtooth &10.0 &82.0 &5.2e-02 &2.2e-02 &5.8e-02 &1.2e-01 &5.2e-02 &1.2e-02 &6.4e-02 &1.1e-02 &1.9e-01 &7.0e-04\\
%	Cu co-lam. &82.0 &82.0 &2.3e-02 &2.3e-02 &1.3e-01 &1.3e-01 &2.3e-02 &1.0e-01 &1.3e-01 &1.1e-02 &8.2e-01 &1.4e-03\\
%    \end{tabular}
%    \caption{
%        These parameters correspond to the measurements from Baglin et al (red), while accounting for the underestimated photon reflectivity (blue).
%        The notation is also used in Eq.~(\ref{eq:k}).
%        $N_i$ and $N_r$ are the photoelectrons per photon on impact and after reflection.
%        $N_\gamma$ is the number of photoelectrons per proton and meter, see Eq.~(\ref{eq:ngamma}).
%    }
%    \label{tab:input_table}
%\end{table}
%
%\subsection{Including scrubbing effects}
%If scrubbing effects are taken into account, the photoemission yield should be lower by roughly 40\%, as supported by the papers presented in Sec.~\ref{sec:Baglin2} and \ref{sec:Cimino}.
