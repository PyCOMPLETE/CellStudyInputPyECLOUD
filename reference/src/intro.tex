
This report describes how photoemission seeding is implemented in PyECLOUD for the electron cloud buildup simulations, especially for the LHC arcs.
It is laid out which quantities are necessary as an input to the simulations and how they have to be specified.
Some of them are derived from the theory of synchrotron radiation, others are material properties taken from published experiment data.
Several papers about measurements for materials used for the LHC beam screens are discussed, in order to arrive at the properties that represent the best knowledge available.
A simulation study that has been performed with these parameters is presented.

Furthermore, possible changes to PyECLOUD are outlined, some of which have already been implemented and could get merged into the main branch of the code.

