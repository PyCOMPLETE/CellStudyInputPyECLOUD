
This report describes how photoemission seeding is implemented in PyECLOUD, especially regarding the electron cloud buildup simulations for the LHC arcs.
It is laid out which quantities are necessary as an input to the simulations and how they have to be specified.
Some of them are derived from the theory of synchrotron radiation, others are material properties taken from published experiment data.
Several papers concerning these measurements are presented and discussed, in order to arrive at those parameters that represent the best knowledge available.

Furthermore, possible changes to PyECLOUD are outlined, some of which have already been implemented and could get merged into the main branch of the code.

